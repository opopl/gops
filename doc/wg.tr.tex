In what follows, we will need here the derivatives 
of $\DPD(i,j)$ with respect to the radius-vectors $\rr_i,\quad i=1,\ldots,N$.
\begin{itemize}
\item $\DPD(i,1),\quad i=3,\ldots,N-2$ 
\begin{align}
  \frac{\partial\DPD(i,1)}{\partial \rr_i}&=2(\rr_i-\rr_{i+1})=2\bb_i\\
  \frac{\partial\DPD(i-1,1)}{\partial \rr_{i}}&=-2\bb_{i-1}
  %\frac{\partial\DPD(i,1)}{\partial \rr_{i+1}}&=-2\bb_i\\
\end{align}
\item $\DPD(i,2),\quad i=3,\ldots,N-2$ 
  \begin{align}
  \frac{\partial\DPD(i,2)}{\partial \rr_i}&=(\rr_{i+1}-\rr_{i+2})=\bb_{i+1}\\
  \frac{\partial\DPD(i-1,2)}{\partial \rr_{i+1}}&=(\rr_{i+1}-\rr_{i+2})=\bb_{i+1}\\
  \frac{\partial\DPD(i-2,2)}{\partial \rr_{i+2}}&=(\rr_{i+1}-\rr_{i+2})=\bb_{i+1}
  \end{align}
\item $\DPD(i,3),\quad i=3,\ldots,N-2$ 
  \begin{align}
  \frac{\partial\DPD(i,3)}{\partial \rr_i}&=(\rr_{i+1}-\rr_{i+2})=\bb_{i+1}\\
  \frac{\partial\DPD(i,3)}{\partial \rr_{i+1}}&=(\rr_{i+1}-\rr_{i+2})=\bb_{i+1}\\
  \frac{\partial\DPD(i,3)}{\partial \rr_{i+2}}&=(\rr_{i+1}-\rr_{i+2})=\bb_{i+1}\\
  \frac{\partial\DPD(i,3)}{\partial \rr_{i+3}}&=(\rr_{i+1}-\rr_{i+2})=\bb_{i+1}
\end{align}
\end{itemize}

\begin{table}[ht]
  \centering
  \caption{$i=2$}
  \begin{tabular}{|p{2cm}|p{3cm}|p{4cm}|p{2cm}|}
\hline
    DP			& $\rr_2$-dependent 	& not-$\rr_2$-dependent & derivative \\
   \hline 
    $(2,1)$ 	& $r_2^2-2\rr_2\rr_{3}$ 	
    		& $r_{3}^2$ 
		& $2\bb_2$ \\
    \hline
    $(2,2)$ 	& $\rr_2\rr_{3}-\rr_2\rr_{4}$  
    		& $\rr_{3}\rr_{4}-r_{3}^2$ 
		& $\bb_{3}$ \\
    \hline
   $(2,3)$		& $\rr_2\rr_{4}-\rr_{2}\rr_{5}$	
    			& $\rr_{3}\rr_{5}-\rr_{3}\rr_{4}$ 
			& $\bb_{4}$ \\
    \hline
    $(1,1)$ 	& $r_{2}^2-2\rr_{2}\rr_{1}$	
    		& $r_{1}^2$ 
		& $-2\bb_{1}$\\
    \hline
    $(1,2)$ 	& $\rr_2\rr_{1}+\rr_2\rr_{3}-\rr_{2}^2$ 
    		& $-\rr_{1}\rr_{3}$ 
    		& $\bb_{1}-\bb_2$ \\
    \hline
   $(1,3)$		& $\rr_2\rr_{4}-\rr_2\rr_{3} $   	
    			& $\rr_{1}\rr_{3}-\rr_{1}\rr_{4}$ 
			& $-\bb_{3}$ \\
\hline
  \end{tabular}
\end{table}

In particular, for 
\begin{enumerate}
  \item $i=1$. 
    \begin{equation}
      \cos\phi_2=\frac
      {\DPD(1,2)\DPD(2,2)-\DPD(2,1)\DPD(1,3)}
      {\sqrt{\DPD(1,1)\DPD(2,1)-\DPD(1,2)^2}}
    \end{equation}
\end{enumerate}

\begin{equation}
  V[\phi]=\sum_{i=2}^{N-2}C_i(1+\cos{\phi_i})+D_i(1+\cos{3\phi_i})
\end{equation}
\begin{equation}
  \v{F}_i=-\frac{\partial V}{\partial \v{r}_i}.
\end{equation}
\begin{enumerate}
  \item $i=1$ Only $\phi_2$ depends on $\v{r}_1$, so
    \begin{equation}
      \frac{\partial V}{\partial \v{r}_1}=
      {\frac{\partial V}{\partial\phi_2}}
      {\frac{\partial\phi_2}{\partial\v{r}_1}}
    \end{equation}
  \item $i=2$
  \item $i=3,\ldots,N-2$
  \item $i=N-1$
  \item $i=N$
\end{enumerate}

Thus, $\XPD(i-2),\XPD(i-1)$ and $\XPD(i)$ depend on $\rr_i$, 
whereas $\XPD(i+j),\ j\ge\ 1$ do not. 
Also, $\phi_{i-1},\ \phi_i,\ \phi_{i+1}$ depends on $\rr_i$. 
\begin{table}
  \centering
  \caption{Radius-vector indices vs. torsional angles}
  \begin{tabular}{|p{3cm}|p{5cm}|}
\hline
$i$ 	& 	torsional angles \\
\hline
  $1$	&	$\phi_2$ \\
  $2$	&	$\phi_2,\ \phi_3$ \\
  $3$	&	$\phi_2,\ \phi_3,\ \phi_4$ \\
  $4$	&	$\phi_2,\ \phi_3,\ \phi_4,\ \phi_5$ \\
  $4,\ldots N-3$	&	$\phi_{i-2},\ \phi_{i-1},\ \phi_i,\ \phi_{i+1}$ \\
  $N-3$		& $\phi_{N-5},\ \phi_{N-4},\ \phi_{N-3},\ \phi_{N-2}$	\\
  $N-2$		& $\phi_{N-4},\ \phi_{N-3},\ \phi_{N-2}$	\\
  $N-1$		& $\phi_{N-3},\ \phi_{N-2}$			\\
  $N$		& $\phi_{N-2}$		\\
  $(1,2,3,4)$	& $\phi_2$ \\
  $(2,3,4,5)$	& $\phi_3$ \\
  $(3,4,5,6)$	& $\phi_4$ \\
  $(4,5,6,7)$	& $\phi_5$ \\
\hline
  \end{tabular}
\end{table}
Thus, the cases $i=1,\ 2,\ 3$ and $i=N,\ N-1,\ N-2$ are special, since the number
of involved torsional angles is less than $4$. 


%17:10:14 (Wed, 25-May-2011):
Then,
\begin{align}
\nn&((\bb_i\times\bb_{i+1}),(\bb_{i+1}\times\bb_{i+2}))=
(\bb_{i+2},((\bb_i\times\bb_{i+1})\times\bb_{i+1}))\\
\nn=&\quad(\bb_{i+2},\bb_{i+1}(\bb_i,\bb_{i+1})-\bb_ib_{i+1}^2)\\
\nn=&\quad(\bb_i,\bb_{i+1})(\bb_{i+1},\bb_{i+2})-b_{i+1}^2(\bb_i,\bb_{i+2})\\
=&\quad\DPD(i)\DPD(i+1)-\DPD(i+1,1)\DPD(i,3).
\end{align}
Thus
\begin{equation}
  \cos\phi_{i}=\frac{\DPD(i)\DPD(i-1)-\DPD(i,1)\DPD(i-1,3)}{\XPD(i)\XPD(i-1)},
  \quad\ i=2,\ldots,N-2.
  %\cos\phi_{i+1}=\frac{\DPD(i)\DPD(i+1)-\DPD(i+1,1)\DPD(i,3)}{\XPD(i)\XPD(i+1)},
  %\quad\ i=1,\ldots,N-3.
  \label{bln-def-tangle}
\end{equation}


%17:14:37 (Wed, 25-May-2011):
Then the product $\XPD(i)\XPD(i+1)$ is expanded as
\begin{align}
  \nn\XPD^2(i)\XPD^2(i+1)&={(\DPD(i,1)\DPD(i+1,1)-\DPD(i)^2)}\\
  \nn&\times{(\DPD(i+1,1)\DPD(i+2,1)-\DPD(i+1)^2)}\\
  \nn&=\DPD(i)^2\DPD(i+1)^2\\
  \nn&+\DPD(i,1)\DPD(i+2,1)\DPD(i+1,1)^2\\
  \nn&-\DPD(i,1)\DPD(i+1,1)\DPD(i+1)^2\\
  &-\DPD(i+1,1)\DPD(i+2,1)\DPD(i)^2
\end{align}

\subsubsection{Torsion angles vs $i, j, k, l$ variables }

\begin{table}
  \centering
  \caption{Torsional angles vs. $i, j, k, l$ variables}
  \begin{tabular}{|*{6}{c|}}
\hline
TA 		& $\rr_{ij}$ 	& $\rr_{kj}$ 	& $\rr_{kl}$	& $\v{m}$	& $\v{n}$	  	\\
\hline
$\phi_2$	& $\bb_1$	& $-\bb_2$ 	& $\bb_3$	& $-\VXPD(1)$ 	& $-\VXPD(2)$ 		\\
$\phi_3$	& $\bb_2$	& $-\bb_3$	& $\bb_4$		\\
$\phi_4$	& $\bb_3$	& $-\bb_4$	& $\bb_5$		\\
$\phi_a$	& $\bb_{a-1}$	& $-\bb_a$	& $\bb_{a+1}$	& $-\VXPD(a-1)$ & $-\VXPD(a)$ 		\\
$\phi_{N-4}$	& $\bb_{N-5}$	& $-\bb_{N-4}$	& $\bb_{N-3}$		\\
$\phi_{N-3}$	& $\bb_{N-4}$	& $-\bb_{N-3}$	& $\bb_{N-2}$		\\
$\phi_{N-2}$	& $\bb_{N-3}$	& $-\bb_{N-2}$	& $\bb_{N-1}$		\\
\hline
  \end{tabular}
\end{table}

\clearpage

