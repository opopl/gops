\chapter{DISCONNECT}

\section{DisconnectionDPS} 
\hypertarget{ddps}
\ddps is a program which purpose is to plot disconnectivity graphs \cite{beckerk97}.

It is adapted to read \ps data files.

Information and options are passed to the program in a file called {\tt dinfo}, which is keyword driven.

\subsection{Compulsory Keywords}

\ksec{DELTA}{dE}
Energetic separation of levels in basin analysis.

\ksec{FIRST}{E1}
Specifies the energy of the highest level on the energy axis.

\ksec{LEVELS}{n}
The number of levels at which to perform the basin analysis.

\ksec{MINIMA}{file}
Specifies filename for minima info.

\ksec{TS}{file}
Specifies filename for transition state info.

\subsection{Optional Keywords}

\ksec{CENTREGMIN}{}
If this keyword is present, then when a node splits into its daughter
nodes, the one containing the global minimum is always placed centrally
(even if other nodes carry more minima). This does not guarantee that
the global minimum is central in the overall diagram because other
nodes may push the one containing the global minimum over to one side.

\ksec{DUMPNUMBERS}{}
If present, a file called {\tt \verb|node_numbers|} is written, listing the minima
associated with each node in each level. Nodes are listed from left to
right within each level.

\ksec{DUMPSIZES}{}
If present, a file called {\tt \verb|node_sizes|} is written, listing how many minima
are represented by each node in each level. Nodes are listed from left to
right in each level.

\ksec{EXCLUDEALL}{}
Removes all minima from the list of minima to be plotted.  This is to be
used in conjunction with the \keyw{PICK} command which can be used to specify
exclusively which minima are to be included.

\ksec{CONNECTMIN}{min}
If present then the analysis for a connected database is based upon minimum
number $min$. If absent then the global minimum is used to judge connectivity.

\ksec{COLOURPRINT}{}
For use with \keyw{TRMIN}, if present colour analysis written to node\_sections.   
Not actually required for colour analysis.

\ksec{IDENTIFY}{}
If present, the branch ends are labelled with the lowest-energy minimum
they represent.

\ksec{IDENTIFY\_NODE}{max\_min}
If present, the nodes are labelled with the format \verb|N1_N2|, where N1 is the number of level,
N2 is the number of the node at that level. The label is only printed if the
number of minima below that node is smaller than \verb|max_min|. With this info
you can pick the number of minima corresponding to that node from the {\tt \verb|node_numbers|} file,
produced by using the keyword \keyw{DUMPNUMBERS}... (and then print any branch of the graph separately)

\ksec{IDENTIFY\_NODE\_SIZE}{max\_min2}
If present, the nodes are labelled with number of minima corresponding to that node. 
The label is only printed if the number of minima below that node is smaller than max\_min2

\ksec{IDMIN}{min}
Label this minimum on the graph. Repeat to label more than one minimum.

\ksec{LABELFORMAT}{fmt}
Specifies the Fortran format string for the energy level labels. The default
is F6.1.

\ksec{LABELSIZE}{n}
Set the size of the fonts in case of the labels (for \keyw{IDENTIFY}, \keyw{IDENTIFY\_NODE} ...)
Default is 10 pt.

\ksec{LETTER}{}
If present, the graph is formatted for American letter paper rather than
European A4.

\ksec{LOWEST}{n}
If present, only the branches leading to the lowest $n$ minima are drawn. The
pruning occurs after the basin analysis, so the "discarded" minima can still
influence the connectivities.

\ksec{MONOTONIC}{}
If present, all minima not lying at the bottom of a monotonic sequences are
not drawn. This tends to reduce the number of branches drastically. If the
keyword \keyw{LOWEST} is also used, the MONOTONIC sequence analysis is applied after
the high energy minima have been discarded.

\ksec{NCONNMIN}{} Minima with NCONNMIN connections or fewer are discarded. Default is zero.

\ksec{NOBARRIERS}{}
If present, all transition state energies are reset to the energy of the higher
of the two minima they connect. This transforms the energy landscape to the
type explored by gmin.

\ksec{PICK}{file}
Specifies the name of a list of numbers of minima, one per line.  Minima on
this list are included on the graph.  Minima preceded with a minus sign are
removed from the graph.  This process is executed after the commands
\keyw{MONOTONIC}, \keyw{LOWEST} and \keyw{EXCLUDEALL} have been executed, thereby making it
possible to override them for particular minima.  Examples: 
\begin{itemize}
\item 1. To remove certain minima from a full plot, just specify PICK 
and a list of negative minima numbers.
\item 2. To include only specific minima, use \keyw{EXCLUDEALL} and
\keyw{PICK} plus a list of positive minima numbers.  All basin analysis includes
the full sample and is performed before minima are removed or added back in.
\end{itemize}

\ksec{NOSPLIT}{}
By default, every minimum is indicated by its own branch, which splits off
from the parent basin even if the minimum and its lowest transition state do
not straddle an energy level in the basin analysis. This is to avoid it being
dependent on precisely where the levels are placed (bulk shifting of the levels
would cause some branch ends to appear or disappear rather than change node
if this were not the case). The \keyw{NOSPLIT} option turns this feature off, so that
if two minima are separated by a barrier lower than the level above their own
energy, they are grouped together. This option should probably never be used.

\ksec{TRMIN}{n max file file ...}
Label $n$ different sections of the graph in colour as specified by the 
minima in each file, one file for each section.  
Each file is a list of numbers of minima, 
one per line as for \keyw{PICK}. $max$ is the total number of minima, not the number 
in the colour files. 
currently used for array allocation.
Colours are chosen automatically to spread over a rainbow spectrum  
(from red to purple) in the order the files are specified but colours can 
be specified individually at both COLOURMARKER in this file. - vkd 

\keyw{TSTHRESH}{threshold} ignore transition states above this threshold.
\keyw{MAXTSENERGY}{threshold} ignore transition states above this threshold.
\keyw{MAXTSBARRIER}{threshold} ignore transition states with both barriers above this threshold.

\keyw{WEIGHTS}{file}

If present, use weights in $file$ to scale the horizontal width. The expected 
format of $file$ is:
bin number  Vmin   Vmax  ln weight

\section{Manipulate}
\hypertarget{manipulate}
\dman is a program for editing disconnectivity graphs produced by \hyperlink{ddps}{\ddps}.
 The user types in commands on the terminal, and the postscript file is
 rewritten after each command.  [Commands are not case sensitive.]

 To get started, make a tree.ps file with \ddps.  Run \prog{ghostview}
 in the background, and start \prog{manipulate}.  Type "READ" to load the graph.
 Now use the coordinates of the mouse pointer in ghostview to identify the
 approximate coordinates $x$ and $y$ of the node you would like to move. \progl{manipulate}
 will find the closest node to the point you specify and can perform the
 following operations.  Nodes are only moved horizontally.

\subsection{Commands}
\ksec{ALIGN}{x y}
Align the node vertically with its parent in the next level up.

\ksec{JOINUP}{x y}
Removes the parent of a node and connects the node to its grandparent
instead.  Only useful for nodes with no sisters.

\ksec{MOVEBY}{x y dx}
Moves the node by a horizontal amount $dx$.

\ksec{MOVETO}{x y x'}
Moves the node to horizontal coordinate $x'$.

\ksec{PIVOT}{x y x' y'}
Moves the $x$ coordinate of a node to $x'$ and pivots all lines connected
below about their intersection with the line $y = y'$. The value of $y'$
should be specified precisely as one of the levels of the graph.

\ksec{PSQUEEZE}{x y f}
Recursively compresses all nodes below the one specified by a fixed
factor $f$ about the $x$ coordinate of the node.

\ksec{QUIT}{} Leave the program

\ksec{RALIGN}{x y}
Recursive align.  Same as \keyw{ALIGN} but translates all nodes connected below
the one moved by the same amount.

\ksec{RMOVEBY}{x y dx}
Recursive moveby.  Same as \keyw{MOVEBY} but translates all nodes connected below
by the same amount.

\ksec{RMOVETO}{x y x'}
Recursive moveto.  Same as \keyw{MOVETO} but translates all nodes below by the
same amount.

\ksec{RSQUEEZE}{x0 x1 y f}
Scales all $x$ coordinates in the range $x0$ to $x1$ on the level at $y$ about the
midpoint of the range by a factor $f$, and recursively translates nodes below
by the same amount.

\ksec{SQUEEZE}{x0 x1 y f}
Same as \keyw{RSQUEEZE} but does not recursively translate the nodes below.

\ksec{UNDO}{} Undoes the last command by reading the previous state from disk.
\ksec{UPALIGN}{x y} Vertically aligns the parent of the specified node.




