\chapter{Models}

\section{Off-lattice BLN model and its modifications}

\subsubsection{Bond vectors}

Bond vectors $\v{b}_i$ are defined as 
\begin{equation}
  \v{b}_i=\v{r}_i-\v{r}_{i+1},\quad i=1,\ldots,N-1,
  \label{bln-def-bv}
\end{equation}
where $\v{r}_i,\quad i=1,\ldots,N$ denote the particle coordinates. We define
a \emph{dot-product array} $\DPD(i,j),\quad i=1,\ldots, N-1;\quad j=1,\ldots,3$ as
\begin{align}
  \nn\DPD(i,1)&=(\v{b}_i,\v{b}_i)\equiv |\v{b}_i|^2\equiv b_i^2=r_i^2+r_{i+1}^2-2(\rr_i,\rr_{i+1}),\\
  \nn\DPD(i,2)&=(\v{b}_i,\v{b}_{i+1})=(\rr_i-\rr_{i+1},\rr_{i+1}-\rr_{i+2})\\
\nn&=\rr_i\rr_{i+1}+\rr_{i+1}\rr_{i+2}-\rr_i\rr_{i+2}-r_{i+1}^2\\
  \DPD(i,3)&=(\v{b}_i,\v{b}_{i+2})=(\rr_i-\rr_{i+1},\rr_{i+2}-\rr_{i+3})\\
\nn&=\rr_i\rr_{i+2}+\rr_{i+1}\rr_{i+3}-\rr_{i}\rr_{i+3}-\rr_{i+1}\rr_{i+2}
  \label{bln-def-dpd}
\end{align}
with the bond vector length $b_i\equiv|\v{b}_i|,\ i=1,\ldots,N-1$, 
and the radius-vector length $r_i\equiv|\v{r}_i|,\ i=1,\ldots,N$.
We note that $\DPD(i,j)$ depends on vectors with indices $i,\ldots,i+j$. Thus,
the following DPs (nine in total) depend explicitly on the radius-vector $\rr_i$, for $i=3,\ldots,N-2$:
\begin{align*}
 \DPD(i,1),&\quad\DPD(i-1,1),\quad\DPD(i,2),\quad\DPD(i-1,2),\quad\DPD(i-2,2),\\ 
 \DPD(i,3),&\quad\DPD(i-1,3),\quad\DPD(i-2,3),\quad\DPD(i-3,3)
\end{align*}
For the special cases $i=1,2$ and $i=N-1,N$, the number of DPs which 
depend on $\rr_i$ is smaller. 
\begin{table}
  \centering
  \begin{tabular}{ccc}
   $i$ 		& Number of DPs		&	DPs \\ 
   $1$		& 3			&	$\DPD(1,1),\DPD(1,2),\DPD(1,3)$ \\
   $2$		& 6			&	$\DPD(2,1),\DPD(1,1),\DPD(2,2),\DPD(1,2),\DPD(2,3),\DPD(1,3)$ \\
   $N-1$	& 6			&	$\DPD(N-1,1),\DPD(N-2,1)$
						$\DPD(N-2,2),\DPD(N-3,2)$,
						$\DPD(N-3,3),\DPD(N-4,3)$	\\	
   $N$		& 3			&	$\DPD(N-1,1),\DPD(N-2,2),\DPD(N-3,3)$ \\
  \end{tabular}
\end{table}
Let us write all the DPs explicitly,
\begin{table}[ht]
  \caption{$i=3,\ldots,N-2$}
  \centering
  \begin{tabular}{|p{2cm}|p{3cm}|p{4cm}|p{2cm}|}
\hline
    DP			& $\rr_i$-dependent 	& not-$\rr_i$-dependent & derivative \\
   \hline 
    $(i,1)$ 	& $r_i^2-2\rr_i\rr_{i+1}$ 	
    		& $r_{i+1}^2$ 
		& $2\bb_i$ \\
    \hline
    $(i-1,1)$ 	& $r_{i}^2-2\rr_{i}\rr_{i-1}$	
    		& $r_{i-1}^2$ 
		& $-2\bb_{i-1}$\\
    \hline
    $(i,2)$ 	& $\rr_i\rr_{i+1}-\rr_i\rr_{i+2}$  
    		& $\rr_{i+1}\rr_{i+2}-r_{i+1}^2$ 
		& $\bb_{i+1}$ \\
    \hline
    $(i-1,2)$ 	& $\rr_i\rr_{i-1}+\rr_i\rr_{i+1}-\rr_{i}^2$ 
    		& $-\rr_{i-1}\rr_{i+1}$ 
    		& $\bb_{i-1}-\bb_i$ \\
    \hline
    $(i-2,2)$ 	& $-\rr_i\rr_{i-2}+\rr_i\rr_{i-1}$ 
    		& $\rr_{i-2}\rr_{i-1}-r_{i-1}^2$ 
		& $-\bb_{i-2}$\\
    \hline
    $(i,3)$		& $\rr_i\rr_{i+2}-\rr_{i}\rr_{i+3}$	
    			& $\rr_{i+1}\rr_{i+3}-\rr_{i+1}\rr_{i+2}$ 
			& $\bb_{i+2}$ \\
    \hline
    $(i-1,3)$		& $\rr_i\rr_{i+2}-\rr_i\rr_{i+1} $   	
    			& $\rr_{i-1}\rr_{i+1}-\rr_{i-1}\rr_{i+2}$ 
			& $-\bb_{i+1}$ \\
    \hline
    $(i-2,3)$		& $\rr_i\rr_{i-2}-\rr_i\rr_{i-1}$ 	
    			& $\rr_{i+1}\rr_{i-2}+\rr_{i-1}\rr_{i+1}$ 
			& $\bb_{i-2}$ \\
    \hline
    $(i-3,3)$		& $-\rr_i\rr_{i-3}+\rr_i\rr_{i-2}$ 
    			& $\rr_{i-3}\rr_{i-1}-\rr_{i-1}\rr_{i-2}$
			& $-\bb_{i-3}$	\\
\hline
  \end{tabular}
\end{table}
\clearpage
\begin{table}[ht]
  \centering
  \caption{$i=1,2,N-1,N$}
  \small
  \begin{tabular}{|p{2cm}|p{4cm}|p{4cm}|p{2.5cm}|}
\hline
    DP			& $\rr_i$-dependent 	& not-$\rr_i$-dependent & derivative \\
\hline
\multicolumn{4}{|c|}{$i=1$} \\ 
   \hline 
    $(1,1)$ 	& $r_1^2-2\rr_1\rr_{2}$ 	
    		& $r_{2}^2$ &
		$2\bb_1$ \\
    \hline
    $(1,2)$ 	& $\rr_1\rr_{2}-\rr_1\rr_{3}$  
    		& $\rr_{2}\rr_{3}-r_{2}^2$ 
		& $\bb_{2}$ \\
    \hline
    $(1,3)$		& $\rr_1\rr_{3}-\rr_{1}\rr_{4}$	
    			& $\rr_{2}\rr_{4}-\rr_{2}\rr_{3}$ 
			& $\bb_{3}$ \\
\hline
\multicolumn{4}{|c|}{$i=2$} \\ 
\hline
    $(2,1)$ 	& $r_2^2-2\rr_2\rr_{3}$ 	
    		& $r_{3}^2$ 
		& $2\bb_2$ \\
    \hline
    $(2,2)$ 	& $\rr_2\rr_{3}-\rr_2\rr_{4}$  
    		& $\rr_{3}\rr_{4}-r_{3}^2$ 
		& $\bb_{3}$ \\
    \hline
   $(2,3)$		& $\rr_2\rr_{4}-\rr_{2}\rr_{5}$	
    			& $\rr_{3}\rr_{5}-\rr_{3}\rr_{4}$ 
			& $\bb_{4}$ \\
    \hline
    $(1,1)$ 	& $r_{2}^2-2\rr_{2}\rr_{1}$	
    		& $r_{1}^2$ 
		& $-2\bb_{1}$\\
    \hline
    $(1,2)$ 	& $\rr_2\rr_{1}+\rr_2\rr_{3}-\rr_{2}^2$ 
    		& $-\rr_{1}\rr_{3}$ 
    		& $\bb_{1}-\bb_2$ \\
    \hline
   $(1,3)$		& $\rr_2\rr_{4}-\rr_2\rr_{3} $   	
    			& $\rr_{1}\rr_{3}-\rr_{1}\rr_{4}$ 
			& $-\bb_{3}$ \\
\hline
\multicolumn{4}{|c|}{$i=N-1$} \\ 
\hline
    $(N-1,1)$ 	& $r_{N-1}^2-2\rr_{N-1}\rr_{N}$ 	
    		& $r_{N}^2$ 
		& $2\bb_{N-1}$ \\
    \hline
    $(N-2,1)$ 	& $r_{N-1}^2-2\rr_{N-1}\rr_{N-2}$	
    		& $r_{N-2}^2$ 
		& $-2\bb_{N-2}$\\
    \hline
    $(N-2,2)$ 	& $\rr_{N-1}\rr_{N-2}+\rr_{N-1}\rr_{N}-\rr_{N-1}^2$ 
    		& $-\rr_{N-2}\rr_{N}$ 
		& $\bb_{N-2}-\bb_{N-1}$ \\
    \hline
    $(N-3,2)$ 	& $-\rr_{N-1}\rr_{N-3}+\rr_{N-1}\rr_{N-2}$ 
    		& $\rr_{N-3}\rr_{N-2}-r_{N-2}^2$ 
		& $-\bb_{N-3}$\\
    \hline
    $(N-3,3)$		& $\rr_{N-1}\rr_{N-3}-\rr_{N-1}\rr_{N-2}$ 	
    			& $\rr_{N}\rr_{N-3}+\rr_{N-2}\rr_{N}$ 
			& $\bb_{N-3}$ \\
    \hline
    $(N-4,3)$		& $-\rr_{N-1}\rr_{N-4}+\rr_{N-1}\rr_{N-3}$ 
    			& $\rr_{N-4}\rr_{N-2}-\rr_{N-2}\rr_{N-3}$
			& $-\bb_{N-4}$	\\
\hline
\multicolumn{4}{|c|}{$i=N$} \\ 
\hline
$(N-1,1)$ 	& $r_{N}^2-2\rr_{N}\rr_{N-1}$	
    		& $r_{N-1}^2$ 
		& $-2\bb_{N-1}$\\
\hline
$(N-2,2)$ 	& $-\rr_N\rr_{N-2}+\rr_N\rr_{N-1}$ 
    		& $\rr_{N-2}\rr_{N-1}-r_{N-1}^2$ 
		& $-\bb_{N-2}$\\
\hline
$(N-3,3)$	& $-\rr_N\rr_{N-3}+\rr_N\rr_{N-2}$ 
    		& $\rr_{N-3}\rr_{N-1}-\rr_{N-1}\rr_{N-2}$
		& $-\bb_{N-3}$	\\
\hline
  \end{tabular}
\end{table}

\subsubsection{Bond angles}

\begin{equation}
  \cos\theta_i=\frac{\DPD(i,2)}{b_ib_{i+1}},\qquad i=2,\ldots,N-1  
  \label{bln-bangle}
\end{equation}

\subsubsection{Torsional angles}

Torsional angle $\phi$ is defined through the dot-product of two normal vectors,
\begin{equation}
  \cos\phi_{i}=-(\v{n}_{i-1},\v{n}_{i}),\quad i=2,\ldots,N-2
\end{equation}
with
\begin{equation}
  %\v{n}_i=\frac{(\bb_i\times\bb_{i+1})}{P_x(i)},\qquad
  %\v{n}_{i+1}=\frac{(\bb_{i+1}\times\bb_{i+2})}{P_x(i+1)},
  \v{n}_{i-1}=\frac{(\bb_{i-1}\times\bb_{i})}{P_x(i-1)},\qquad
  \v{n}_{i}=\frac{(\bb_{i}\times\bb_{i+1})}{P_x(i)},
\end{equation}
where $\XPD(i)\equiv|(\bb_i\times\bb_{i+1})|$.
Then,
\begin{align}
\nn&((\bb_i\times\bb_{i+1}),(\bb_{i+1}\times\bb_{i+2}))=
(\bb_{i+2},((\bb_i\times\bb_{i+1})\times\bb_{i+1}))\\
\nn=&\quad(\bb_{i+2},\bb_{i+1}(\bb_i,\bb_{i+1})-\bb_ib_{i+1}^2)\\
\nn=&\quad(\bb_i,\bb_{i+1})(\bb_{i+1},\bb_{i+2})-b_{i+1}^2(\bb_i,\bb_{i+2})\\
=&\quad\DPD(i,2)\DPD(i+1,2)-\DPD(i+1,1)\DPD(i,3).
\end{align}
Thus
\begin{equation}
  %\cos\phi_{i+1}=\frac{\DPD(i,2)\DPD(i+1,2)-\DPD(i+1,1)\DPD(i,3)}{\XPD(i)\XPD(i+1)},
  %\quad\ i=1,\ldots,N-3.
  \cos\phi_{i}=\frac{\DPD(i,2)\DPD(i-1,2)-\DPD(i,1)\DPD(i-1,3)}{\XPD(i)\XPD(i-1)},
  \quad\ i=2,\ldots,N-2.
  \label{bln-def-tangle}
\end{equation}
It is possible to obtain $\XPD$ through $\DPD$:
\begin{equation}
  \XPD^2(i)\equiv\ (\bb_i\times\bb_{i+1})^2
  =b_i^2b_{i+1}^2-(\bb_i,\bb_{i+1})^2
  =\DPD(i,1)\DPD(i+1,1)-\DPD(i,2)^2.
\end{equation}
Then the product $\XPD(i)\XPD(i+1)$ is expanded as
\begin{align}
  \nn\XPD^2(i)\XPD^2(i+1)&={(\DPD(i,1)\DPD(i+1,1)-\DPD(i,2)^2)}\\
  \nn&\times{(\DPD(i+1,1)\DPD(i+2,1)-\DPD(i+1,2)^2)}\\
  \nn&=\DPD(i,2)^2\DPD(i+1,2)^2\\
  \nn&+\DPD(i,1)\DPD(i+2,1)\DPD(i+1,1)^2\\
  \nn&-\DPD(i,1)\DPD(i+1,1)\DPD(i+1,2)^2\\
  &-\DPD(i+1,1)\DPD(i+2,1)\DPD(i,2)^2
\end{align}
Thus, $\XPD(i-2),\XPD(i-1)$ and $\XPD(i)$ depend on $\rr_i$, 
whereas $\XPD(i+j),\ j\ge\ 1$ does not. 
Also, $\phi_{i-1},\ \phi_i,\ \phi_{i+1}$ depends on $\rr_i$. 
\begin{table}
  \centering
  \caption{Radius-vector indices vs. torsional angles}
  \begin{tabular}{|p{3cm}|p{5cm}|}
\hline
$i$ 	& 	torsional angles \\
\hline
  $1$	&	$\phi_2$ \\
  $2$	&	$\phi_2,\ \phi_3$ \\
  $3$	&	$\phi_2,\ \phi_3,\ \phi_4$ \\
  $4$	&	$\phi_2,\ \phi_3,\ \phi_4,\ \phi_5$ \\
  $4,\ldots N-3$	&	$\phi_{i-2},\ \phi_{i-1},\ \phi_i,\ \phi_{i+1}$ \\
  $N-3$		& $\phi_{N-5},\ \phi_{N-4},\ \phi_{N-3},\ \phi_{N-2}$	\\
  $N-2$		& $\phi_{N-4},\ \phi_{N-3},\ \phi_{N-2}$	\\
  $N-1$		& $\phi_{N-3},\ \phi_{N-2}$			\\
  $N$		& $\phi_{N-2}$		\\
  $(1,2,3,4)$	& $\phi_2$ \\
  $(2,3,4,5)$	& $\phi_3$ \\
  $(3,4,5,6)$	& $\phi_4$ \\
  $(4,5,6,7)$	& $\phi_5$ \\
\hline
  \end{tabular}
\end{table}
Thus, the cases $i=1,\ 2,\ 3$ and $i=N,\ N-1,\ N-2$ are special, since the number
of involved torsional angles is less than $4$. 

\subsubsection{Bekker formulas}

We will also need \cite{Bekker95}
\begin{align}
  \Gi\quad=&\quad \frac{dV(\phi)}{d\phi}r_{kj}\frac{\v{m}}{|\v{m}|^2}\\
  \Gl\quad=&\quad-\frac{dV(\phi)}{d\phi}r_{kj}\frac{\v{n}}{|\v{n}|^2}\\
\Gj\quad=&\quad-\Gi +\left(\frac{\v{r}_{ij}\cdot{\v{r}_{kj}}}{r_{kj}^2}\right)\Gi
  -\left(\frac{\v{r}_{kl}\cdot{\v{r}_{kj}}}{r_{kj}^2}\right)\Gl\\
  \Gk\quad=&\quad-\Gl
  -\left(\frac{\v{r}_{ij}\cdot{\v{r}_{kj}}}{r_{kj}^2}\right)\Gi
  +\left(\frac{\v{r}_{kl}\cdot{\v{r}_{kj}}}{r_{kj}^2}\right)\Gl\\
\v{m}\quad\equiv&\quad\v{r}_{ij}\times\v{r}_{kj}\\
\v{n}\quad\equiv&\quad\v{r}_{kj}\times\v{r}_{kl}
\end{align}

\subsubsection{Gradients vs torsional angles}

\begin{align}
  \nn\GG{1}=&\Gi(\phi_2)\\
  \nn\GG{2}=&\Gi(\phi_3)+\Gj(\phi_2)\\
  \nn\GG{3}=&\Gi(\phi_4)+\Gj(\phi_3)+\Gk(\phi_2)\\
  \nn\GG{4}=&\Gi(\phi_5)+\Gj(\phi_4)+\Gk(\phi_3)+\Gl(\phi_2)\\
  \nn\ldots\\
  \nn\GG{i}=&\Gi(\phi_{i+1})+\Gj(\phi_i)+\Gk(\phi_{i-1})+\Gl(\phi_{i-2}),\quad i=4,\ldots,N-3\\
  \nn\ldots\\
  \nn\GG{N-3}=&\Gi(\phi_{N-2})+\Gj(\phi_{N-3})+\Gk(\phi_{N-4})+\Gl(\phi_{N-5})\\
  \nn\GG{N-2}=&\Gj(\phi_{N-2})+\Gk(\phi_{N-3})+\Gl(\phi_{N-4})\\
  \nn\GG{N-1}=&\Gk(\phi_{N-2})+\Gl(\phi_{N-3})\\
  \GG{N}=&\Gl(\phi_{N-2})
\end{align}

\subsubsection{Gradients for the BLN model}

For $V(\phi)=A(1+\cos\phi)+B(1+\cos{3\phi})$, this yields:
\begin{itemize}
\item $i=1$ 
\begin{align}
  \nn\GG(\rr_1)=& -(-A\sin\phi_2-3B\sin{3\phi_2})
  b_2\frac{\bb_1\times\bb_2}{|\bb_1\times\bb_2|^2}\\
  =&\quad\VXPD(1)\frac{b_2(A\sin\phi_2+3B\sin(3\phi_2))}{\XPD(1)^2}
\end{align}
\item $i=2$. We have
  \begin{align}
    \Gi(\phi_3)&=\quad\VXPD(2)\frac{b_3(A\sin\phi_3+3B\sin(3\phi_3))}{\XPD(2)^2}\\
    \Gj(\phi_2)&=\quad -\Gi(\phi_2)+
  \end{align}
\end{itemize}

\subsection{Determination of the global minimum}


